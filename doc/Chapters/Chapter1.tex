% Chapter 1

\chapter{Introduction}  % Main chapter title

\label{Chapter1}  % For referencing the chapter elsewhere, use \ref{Chapter1}

\lhead{Chapter 1. \emph{Introduction}}  % This is for the header on each page - perhaps a shortened title

%----------------------------------------------------------------------------------------

This section provides an overview of what IM3D can and can't do, describe what it means for IM3D to be an open-source code.
%, and acknowledges the funding, citations and people who have contributed to IM3D during its developing.

%----------------------------------------------------------------------------------------

\section{What is IM3D}

IM3D is an open-source parallel 3D Monte Carlo (MC) code for rapidly simulating the transportation of ions and the production of defects in nanostructured materials. It is an accurate, efficient and universal 3D version of MC model developed based on the standard SRIM databases\cite{Ziegler:2010}, the fast database indexing technique\cite{Schiettekatte:2008} and MPI parallel algorithm as well as the 3D structural algorithms of Constructive Solid Geometry (CSG) / Finite Element Triangulated Mesh (FETM) methods\cite{Li:2005,Li:2008,Li:2009,Li2:2011,Zhang:2011,Li:2013}. It can model arbitrary-complex 3D targets made of different geometric elements each of which with different materials. Both the 3D distribution of ions and also all kinetic phenomena associated with the ion’s energy loss, i.e., amorphization, damage, sputtering, ionization and phonon production, can be calculated by IM3D with following all target atom cascades in detail. Thus, IM3D code provides a general and robust theoretical approach to analysis the effects in primary damage processes and the corresponding 3D space-distributions of primary defects in nanostructured materials under ion beam irradiation.
%including electronic and nuclear energy depositions, back-scatted/implanted ions, $dpa$, interstitials, vacancies and sputtering atoms, etc.

The development of IM3D is mainly includes three aspects, i.e. the accurate physical models, the universal 3D structural models and the efficient calculation algorithms. The physical parameters used in the code, such as, the electronic stopping power and energy straggling parameters, are generated from SRMModule.exe provided by SRIM package\cite{Ziegler:2010}. The 3D nanostructured samples can be generated by graphical softwares beforehand and traced by the sophisticated 3D structural algorithms based on the CSG/FETM methods\cite{Li:2005,Li:2008,Li:2009,Li2:2011,Zhang:2011,Li:2013}. In order to further increasing the efficiency of the code, we introduced the fast database indexing technique proposed in Corteo\cite{Schiettekatte:2008} to sampling the scattering and azimuthal angles as well as a linear speed-up MPI parallel algorithm or a multi-threading parallel algorithm. Detailed description of the physical basement can be found in our papers or in Appendix A.

%The ion transportation in IM3D is treated by considering an idealization of the technique principles where each %incoming ion slows down in straight line in the material until it eventually makes a collision during which it is scattered, %produces a recoil or emits a particle in a selected direction. The emitted ion also slows down along a straight trajectory %until it leaves the sample. In this idealization, the energy loss is related to the stopping power and the kinematics of the %scattering/recoiling collision. The computation of ion slowdown is carried out under three important approximations, i.e., %the binary collision approximation (BCA), the central potential approximation (CPA), and the random phase %approximation (RPA). For each collision cycle, the steps will be the determination of the flight length, the impact %parameter, the collision partner, the scattering and azimuthal angles, and, based on the latter two, of the new flying %direction. Then, the energy loss is computed, the position is incremented from the last collision locus to the new one %according to the flight length and the new flying direction, and the cycle restarts from the new collision locus until the %ion comes to rest or leaves the target. Trajectories computation helps to account as precisely as possible for the effect %of multiple collisions by reproducing in details the deviations from straight line trajectories.

IM3D can run efficiently on different platforms, including not only single- or multi-processors desktop or laptop machines but also parallel computers. Simultaneous multi-threading technique has be included in IM3D code to run on a multi-processors system. It can also run on any parallel machine that compiles plain C and supports the MPI message-passing library.

%IM3D code is developed to be an open-source code in terms of the GNU general public license[91], except for the parts of CSG/FETM models whose copyrights belong to Prof. Z.J. Ding. Thus, in order to part rights reserved, the routines related to CSG/FETM models are compiled to static libraries in IM3D package.

IM3D is a freely-available open-source code except for the geometric modules (copyrights belong to Prof. \href{http://micro.ustc.edu.cn/Members/zjding.htm}{Zejun Ding}), distributed under the terms of the GNU Public License, which means you can use or modify the code however you wish but commercial purposes. In order to part rights reserved, the routines related to CSG/FETM models are compiled to static libraries in IM3D package. In addition, a part of the modules in IM3D refer to the open-source codes, \href{http://www.nano.uni-jena.de/Forschung/Physik+mit+Ionenstrahlen/iradina-p-103.html}{Iradina}\cite{Borschel:2011} and \href{http://www.lps.umontreal.ca/~schiette/index.php?n=Recherche.Corteo}{Corteo}\cite{Schiettekatte:2008}.

%IM3D was sponsored by Prof. \href{http://li.mit.edu}{Ju Li} and developed by Dr. \href{}{Yonggang Li} when his visiting %to MIT in 2014, with the fundings mainly come from CAS and partly from MIT. Useful discussions and contributions %from Dr. Machal Short  and Yang Yang are very appreciate.

%In the most general sense, IM3D is an accurate, efficient and universal 3D Monte Carlo model developed based on the standard SRIM databases, the fast database indexing technique and MPI parallel algorithm as well as the 3D structural algorithms of CSG/FETM methods, which can provide a general and robust theoretical approach to analysis the effects in primary damage processes and the corresponding 3D space-distributions of primary defects in nanostructured materials under ion beam irradiation.

%----------------------------------------------------------------------------------------

\section{IM3D Features}

This section highlights IM3D features: %with pointers to the following sections which give more details.

\begin{itemize}

\item open-source distribution with highly portable C;
\item ion with atomic number of $1-92$ and energy of $10~eV-2~GeV/amu$, as well as different ion beam shape distribution, i.e., random, centered, defined position, random square around predefined position, Gaussian beam and etc.
\item arbitrary complex targets constructed by the 3D geometric algorithms of CSG/FETM methods\cite{Li:2005,Li:2008,Li:2009,Li2:2011,Zhang:2011,Li:2013}, with complex materials including single elements ($1-92$), alloys and compounds;
\item generate input shapes in the form of different formats (e.g. $opengl$, $ply2$ and etc.) with different standard finite element softwares, e.g. \href{http://geuz.org/gmsh/}{Gmsh}\cite{Geuzaine:2009}, \href{https://cubit.sandia.gov}{Cubit} and etc.;
\item 1D (bulk and multi-layers) or 3D systems with or without semi-infinite substrate;
\item runs from an input script or four separate input files (i.e., temp\_compfile.im3d, temp\_configfile.im3d, temp\_matfile.im3d and temp\_structfile.im3d);
\item runs on a single processor or in parallel with distributed-memory message-passing parallelism (MPI);
\item electronic energy loss and straggling are based on the standard \href{http://www.srim.org}{SRIM} databases\cite{Ziegler:2010}, and Bragg's rule\cite{Bragg:1905} is used to estimate the stopping power of a compound by the linear combination of the stopping powers of its individual elements;;
\item uses fast database indexing technique (see \href{http://www.lps.umontreal.ca/~schiette/index.php?n=Recherche.Corteo}{Corteo}\cite{Schiettekatte:2008}) or the MAGIC approximation formula \cite{Ziegler:2010} for sampling in terms of accuracy and efficiency;
\item uses the analytical modified Kinchin-Pease (KP) model\cite{Kinchin:1955,Norgett:1975} or the computationally full cascade (FC) simulation for defect generation processes;
\item uses a screened repulsive Coulomb potential described by a dimensionless screening function, such as the Thomas-Feimi potential\cite{Sommerfeld:1932}, the Lenz-Jensen potential\cite{Lenz:1932}, the Moliere potential\cite{Moliere:1947}, the Bohr potential\cite{Bohr:1948} and the universal Ziggler-Biersack-Littmark (ZBL) potential\cite{Ziegler:2010} to describe the interaction potential between two atoms;
\item output primary damage information including 1D (depth) and 3D distributions of electronic and nuclear energy depositions, back-scattering/implanted ions, $dpa$, interstitials, vacancies and sputtering atoms, etc.
\item the output distribution files are in the format of $.cfg$, $.msh$ or $.vtk$, which can be viewed by various pre- and post-processing tools such as \href{http://li.mit.edu/Archive/Graphics/A/}{AtomEye}\cite{Li:2003}, \href{http://geuz.org/gmsh/}{Gmsh}\cite{Geuzaine:2009}, \href{http://www.paraview.org/}{ParaView}, \href{https://cubit.sandia.gov/}{Cubit} and etc.

\end{itemize}

%----------------------------------------------------------------------------------------

\section{IM3D Non-features}

IM3D is designed to efficiently simulate the tracing of ions in static arbitrary complex 3D systems. Some features that IM3D does not yet (maybe in the future version) support are list below:

\begin{itemize}

\item real-time tracing plot;
\item restart;
\item 64-bit system compatibility;
\item dynamic version.

\end{itemize}

The serial version of the code can output xyz-coordinates of ions/atoms, which can give the plot of the tracing trajectories after the simulation. In the code, we introduced the fast database indexing technique (see \href{http://www.lps.umontreal.ca/~schiette/index.php?n=Recherche.Corteo}{Corteo}\cite{Schiettekatte:2008}) which is mainly based on a 32-bit system. Thus, -m32 or -arch i386 should be used in the makefile when compiling. We will also introduced a 64-bit version in the future. IM3D is a static version of the system without change the geometry shape of the system during irradiation.

Otherwise, many of the tools needed to pre- and post-process the data for 3D geometry shapes are not included in the IM3D kernel. Specifically, IM3D itself does not:

\begin{itemize}

\item run thru a GUI (would be included in the future version);
\item build 3D geometry structures;
\item perform sophisticated analyses;
\item visualize simulation results;
\item plot output data.

\end{itemize}

A few tool for constructing the input 3D geometry structures are provided as part of the IM3D package, as described in section X. Although users can be write their own tools for these tasks, but we recommend to use our tool to generate the CSG format geometry structures and the open-source software \href{http://geuz.org/gmsh/}{Gmsh} to generate the FETM format geometry structures. For high-quality visualization we recommend the \href{http://li.mit.edu/Archive/Graphics/A/}{AtomEye}\cite{Li:2003}, \href{http://geuz.org/gmsh/}{Gmsh}\cite{Geuzaine:2009} or \href{http://www.paraview.org/}{ParaView} softwares.

%----------------------------------------------------------------------------------------

\section{Open-source Distribution}

IM3D comes with no warranty of any kind. It is a copyrighted code that is distributed free-of-charge, under the terms of the \href{}{GNU Public License} (GPL). This is often referred to as open-source distribution - see \href{www.gnu.org}{www.gnu.org} or \href{www.opensource.org}{www.opensource.org} for more details. The legal text of the GPL is in the LICENSE file that is included in the IM3D distribution.

Here is a summary of what the GPL means for IM3D users:

(1) Anyone is free to use, modify, or extend IM3D in any way they choose, without including for commercial purposes.

(2) If you distribute a modified version of IM3D, it must remain open-source, meaning you distribute it under the terms of the GPL. You should clearly annotate such a code as a derivative version of IM3D.

(3) If you release any code that includes IM3D source code, then it must also be open-sourced, meaning you distribute it under the terms of the GPL.

(4) If you give IM3D files to someone else, the GPL LICENSE file and source file headers (including the copyright and GPL notices) should remain part of the code.

In the spirit of an open-source code, these are various ways you can contribute to making IM3D better. Any questions please send email to \href{ygli@theory.issp.ac.cn}{Yonggang Li (Y.G. Li)} .

%----------------------------------------------------------------------------------------

\section{Citations}

Please cite our paper if you use subroutines in this package. Thanks.

Y.G. Li, Y. Yang, M. Short, Z.J. Ding, Z. Zeng and J. Li, Fast simulation of primary damages in arbitrarily complexed nanostructured materials under ion irradiation, to be published;

Y.G. Li, Y. Yang, M. Short, Z.J. Ding, Z. Zeng and J. Li, IM3D: A 3D Parallel Monte Carlo Simulation Code for Ion Irradiation of Nanostructured Materials, to be published.

%----------------------------------------------------------------------------------------

